\documentclass[conference]{IEEEtran}
\IEEEoverridecommandlockouts
% The preceding line is only needed to identify funding in the first footnote. If that is unneeded, please comment it out.
\usepackage[portuges,brazil,english]{babel}
\usepackage[utf8]{inputenc}
\usepackage{amsmath,amssymb,amsfonts}
\usepackage{algorithmic}
\usepackage{graphicx}
\usepackage{textcomp}
\usepackage{float}
\def\BibTeX{{\rm B\kern-.05em{\sc i\kern-.025em b}\kern-.08em
    T\kern-.1667em\lower.7ex\hbox{E}\kern-.125emX}}

\usepackage{subcaption}
\usepackage[style=mla,guessmedium=false,backend=bibtex]{biblatex}
\addbibresource{report.bib}

\begin{document}

\renewcommand{\figurename}{Fig.}
\renewcommand{\refname}{Referências}

\title{Técnicas de Clusterização e Visualização de Dados}

\author{\IEEEauthorblockN{Carolina F. Cuba}
\IEEEauthorblockA{
226004 \\
carolinacuba23@gmail.com}
\and
\IEEEauthorblockN{Leonardo de Melo Joao}
\IEEEauthorblockA{
228118 \\
l228118@g.unicamp.br}}

\maketitle

\section{Introdução}	
	
	
\section{K-means}
	\subsection{Definindo o valor do k}
	Foram utilizadas duas métricas para definir o melhor valor de k: (1) coeficiente de silhueta; (2) curva do cotovelo.
	Com $80%$ dos dados, os algorítmos encontraram um minimo local e um máximo local na curva de cotovelo e coeficiente de silhueta respectivamente, procurando possíveis $k$ com passo de 10. Após alguns passos os algoritmos saem desses extremos locais e convergem sempre para resultados melhores conforme o número de clusters aumenta. Obviamente, com apenas 1 ou 2 elementos por cluster, essas métricas obteriam resultados muito satisfatórios, mas o propósito de clusterização seria completamente perdido. Nesse cenário, foram executados testes com passo 1 para determinar exatamente qual era o melhor valor encontrados nos extremos locais.
	O mínimo encontrado na curva de cotovelo foi de $0.663$ para $k = 77$, enquanto o valor máximo encontrado no coeficiênte de silhueta foi de $0.047$ para $k = 76$. Ambos resultados são ruins, porém decidimos analisá-los 

\printbibliography

\end{document}
